\chapter{Continuity}
\subsubsection*{Proving x^{}2 is continuous but NOT uniformly continuous on (-inf, inf)}
We could show that $f(x)=x^{2}$ is not uniformly continous on $(-\infty, \infty)
= \real$ using the following theorem:

\begin{thm}
  \label{unbounded-derivative-implies-not-uniformly-continous} $\lim_{x \to
    \infty}f'(x) = \infty$ $\implies$ $f$ is not uniformly continous.
\end{thm}
We have that $f'(x)=2x$ is not bounded. But we will not use (\ref{unbounded-derivative-implies-not-uniformly-continous})
here.

\begin{mydef}
  \label{definition-of-continouity} $f$ is continuous on $I$ means
  \begin{equation}
    \forall a \in I, \forall \epsilon > 0, \exists \delta > 0, \forall x \in I :
    | x-a | < \delta \implies |f(x) - f(a)| < \epsilon
  \end{equation}
\end{mydef}

\begin{mydef}
  \label{definition-of-uniform-continouity} $f$ is uniformly continous on $I$ means
  \begin{equation}
    \forall \epsilon > 0, \exists \delta > 0, \forall x_{1}, x_{2} \in I: |x_{1}-
    x_{2}| < \delta \implies |f(x_{1}) - f(x_{2})| < \epsilon
  \end{equation}
\end{mydef}

\begin{mydef}
  \label{definition-of-non-uniform-continouity} $f$ is not uniformly continous on
  $I$ means
  \begin{equation}
    \label{equation-of-non-uniform-continouity}
    \begin{aligned}
      \exists \epsilon > 0, \forall \delta > 0, \exists x_{1}, x_{2} \in I: & |x_{1} - x_{2}| < \delta           \\
      \text{ ``but" \slash $\,$ ``and at the same time"}:                   & |f(x_{1})- f(x_{2})| \geq \epsilon
    \end{aligned}
  \end{equation}
  One can use predicate logic to show that this is the negation of the Definition
  \ref{definition-of-uniform-continouity}.
\end{mydef}

With these definitions we want to show that $x^{2}$ is uniformly continous on $[0
,1]$ but not uniformly continous on $\real$

\begin{thm}
  $f(x)=x^{2}$ is uniformly continous on $[0,1]$.
\end{thm}
\begin{proof}
  Given $\epsilon > 0$. Lets choose a $\delta>0$ we don't know yet. So $\delta :\equiv
  \boxed{\color{white}\epsilon/2}$. Let $x_{1}, x_{2} \in [0,1]$. Suppose $|x_{1}
  -x_{2}| < \delta$. We can see that
  \begin{equation}
    \label{expanded-x1pow2-minus-x2pow2}|x_{1}^{2} - x_{2}^{2}| = |(x_{1} - x_{2}
    )(x_{1}+x_{2})| = |x_{1} - x_{2}||x_{1}+x_{2}| < \delta |x_{1}+x_{2}| \leq \delta
    2
  \end{equation}
  So for a given $\epsilon$ we choose $\delta :\equiv \boxed{\epsilon/2}$. \\ Then
  we have $|(f(x_{1})-f(x_{2})| \leq |x_{1}^{2} - x_{2}^{2}| \leq 2\cdot \delta =
  \epsilon \quad \forall x_{1}, x_{2} \in [0,1]$.\\ So $f(x)=x^{2}$ is uniformly
  continuous on $[0,1]$.
\end{proof}

\begin{thm}
  $f(x)=x^{2}$ is not uniformly continuous on $\real$.
\end{thm}
\begin{proof}
  Looking at the definition \ref{definition-of-non-uniform-continouity} with it's
  equation (\ref{equation-of-non-uniform-continouity}) we can choose $\epsilon>0$
  as well as $x_{1}, x_{2} \in \real$ freely. So let $\epsilon : \equiv 1$. So given
  any $\delta > 0$ we have to pick $x_{1}, x_{2} \in \real$ such that $|x_{1}-x_{2}
  |<\delta$ and at the same time $|x_{1}^{2}-x_{2}^{2}|\geq1=\epsilon$. So let
  again $|x_{1}^{2}-x_{2}^{2}|=|x_{1}^{2}-x_{2}^{2}||x_{1}^{2}+x_{2}^{2}|$ like
  in (\ref*{expanded-x1pow2-minus-x2pow2}) . If we pick
  $x_{1} :\equiv\frac{\delta}{2}, x_{2} :\equiv 0$ we will end up with
  $|x_{1}^{2}-x_{2}^{2}|= \delta^{2}/4´$ which will not be $\geq 1$ for any
  given $\delta>0$. Similarly, if we pick
  $x_{1} : \equiv \delta + \frac{\delta}{2}, x_{2} :\equiv \delta$ we will end
  up with $|x_{1}^{2}-x_{2}^{2}|= (\delta/2)((5\delta)/2)=(5\delta^{2})/2$ which
  will not be $\geq 1$ for any given $\delta>0$. Finally, if we pick
  $x_{1} : \equiv \frac{1}{\delta}+ \frac{\delta}{2}\in \real, x_{2} :\equiv \frac{1}{\delta}
  \in \real$
  we will end up with
  $|x_{1}^{2}-x_{2}^{2}|= \frac{\delta}{2}(\frac{2}{\delta}+\frac{\delta}{2})=1+\delta
  ^{2}/4$
  which will be $\geq 1=\epsilon$ for any given $\delta>0$. So $f(x)=x^{2}$ is not
  uniformly continuous on $\real$
\end{proof}

\pagebreak

\begin{thm}
  $f(x)=x^{2}$ is continuous on $\real$
\end{thm}
\begin{proof}
  We must show that $\lim_{x \to a}x^{2} = a^{2} \quad \forall a\in \real$.
  Given $\epsilon >0, a\in \real.$ \\ Choose
  $\delta = \boxed{\color{white}\min \left\{1, \frac{\epsilon}{1+2|a|}\right\}}$
  we dont know yet.

  Let $x\in \real$ be arbitrary. Suppose $0<|x-a|<\delta$. Then
  we have
  \begin{dmath}
    |x^2-a^2|=|x-a||x+a|<\delta|x+a|.
  \end{dmath}
  If we would choose $\delta$ to be $\leq 1$ to begin with, we would have
  \begin{dmath}
    |x+a| = |x-a+2a| \leq |x-a| + |2a| < 1 + |2a´|.
  \end{dmath}
  So we have
  \begin{equation}
    |x^{2}-a^{2}| < \delta (1+2|a|)
  \end{equation}
  If we choose $\delta = \boxed{\min \left\{1, \frac{\epsilon}{1+2|a|}\right\}}$,
  we have
  \begin{dmath}
    |x^2-a^2| < \delta (1+2|a|) \leq \frac{\epsilon}{1+2|a|} (1+2|a|) = \epsilon
  \end{dmath}
\end{proof}