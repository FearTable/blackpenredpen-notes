\chapter{Basics}
\section{Polynomials}
\subsection*{The cover-up method \& why it works! (for partial fractions decomposition)}
\url{https://www.youtube.com/watch?v=fgPviiv_oZs}

\begin{equation}
  \frac{2x-1}{(x-1)(x-2)(x-3)}=\frac{A}{x-1}+\frac{B}{x-2}+\frac{C}{x-3}
\end{equation}

We have

\begin{equation}
  \begin{aligned}
    A &= \frac{2(1)}{(1-2)(1-3)}   = \frac{1}{2} \\
    B &= \frac{2(2)-1}{(4-1)(2-3)} = -3 \\´
    C &= \frac{2(3)-1}{(3-1)(3-2)} = \frac{5}{2}
  \end{aligned}
\end{equation}

And therefore

\begin{equation}
  \frac{2x-1}{(x-1)(x-2)(x-3)}=\frac{\frac{1}{2}}{x-1}+\frac{-3}{x-2}+\frac{\frac{5}{2}}{x-3}
\end{equation}

It works because

\begin{equation}
  \frac{2x-1}{(x-2)(x-3)}=A+\frac{B}{x-2}(x-1)+\frac{C}{x-3}(x-1)
\end{equation}