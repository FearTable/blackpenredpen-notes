\chapter{Logarithms}
\subsubsection*{Is this equation solvable? x\^{}ln(4)+x\^{}ln(10)=x\^{}ln(25)}
Let us look at the equation
\begin{equation}
	\label{x-to-ln-power-equation}
	x^{\ln 4 } + x^{\ln 10} = x^{\ln 25 },
\end{equation}
which has the trivial solution $x=0$. Note that
\begin{equation}
	a^{\log_b c} = \left( b ^{\log_{b} a} \right) ^ {\log _b c} = b^{\log_b a \cdot \log_b c} = \left( b^{\log_b c }\right) ^{\log_b a} = c^{\log_b a}
\end{equation}

Therefore, for $x \neq 0$, (\ref{x-to-ln-power-equation}) becomes
\begin{align}
	4^{\ln x} + 10^{\ln x}& = 25^{\ln x} \\
	1 + 
	\frac{10^{\ln x}}{4^{\ln x}} & = 
	\frac{25^{\ln x}}{4^{\ln x}} \\
	1 + \left( \frac{5}{2} \right)^{\ln x} &=
	\left( \left(  \frac{5}{2} \right)^2 \right)^{\ln x} \\
	\left( \left(  \frac{5}{2} \right)^{\ln x} \right)^2
	-\left( \frac{5}{2} \right)^{\ln x} -1 &= 0
\end{align}

Therefore, using the midnight formula, we have
\begin{equation}
	\left( \frac{5}{2} \right)^{\ln x}=\frac{-(-1)\pm \sqrt{(-1)^2-4(1)(-1)}}{2(1)}=\frac{1 + \sqrt{5}}{2} \equiv \varphi,
\end{equation}
where $\varphi$ is the golden ratio. So
\begin{align}
	\log_{\frac{5}{2}} \left( \frac{5}{2} \right)^{\ln x} & =
	\log_{\frac{5}{2}} \varphi \\
	\ln x &= \log_{\frac{5}{2}} \varphi \\
	e^{\ln x} &= e^{\log_{\frac{5}{2}}\varphi} \\
	x &= e^{\log_{\frac{5}{2}}\varphi} \approx 1.691
\end{align}
